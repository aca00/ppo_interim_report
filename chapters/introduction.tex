\chapter{Introduction}
Plug-ins are small applications that allow you to view certain types of content within your web browser. Common plug-ins include Adobe Reader, which lets you view PDF files in your browser; and Micosoft Silverlight, which is often required for video sites like Netflix.The browser plug-in focuses on bringing the active users of a  website to collaborate on sharing information and helping the users reach the space that provides them with the desired content by ensuring their privacy and safety.the browser plug-in could help us to use websites in the moist efficient way.Here we have introduced the more advanced version of the web browser plug-in where you can chat with live users to share information, chat and the most exciting and the novel feature of rating and make reviews for the website.Plug-ins first gained popularity in the 1990s as software and microprocessors became more powerful. One of the first programs to make extensive use of plug-ins was Adobe Photoshop, an image-processing and editing program. Early plug-ins provided enhanced functions such as special effects, filters, and other options for manipulating images within Photoshop.


\section{Scope and objectives}

\section{problem statements} 
